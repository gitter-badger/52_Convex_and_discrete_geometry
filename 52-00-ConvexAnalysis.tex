\documentclass[12pt]{article}
\usepackage{pmmeta}
\pmcanonicalname{ConvexAnalysis}
\pmcreated{2013-03-22 15:11:47}
\pmmodified{2013-03-22 15:11:47}
\pmowner{matte}{1858}
\pmmodifier{matte}{1858}
\pmtitle{convex analysis}
\pmrecord{7}{36954}
\pmprivacy{1}
\pmauthor{matte}{1858}
\pmtype{Topic}
\pmcomment{trigger rebuild}
\pmclassification{msc}{52-00}

\endmetadata

% this is the default PlanetMath preamble.  as your knowledge
% of TeX increases, you will probably want to edit this, but
% it should be fine as is for beginners.

% almost certainly you want these
\usepackage{amssymb}
\usepackage{amsmath}
\usepackage{amsfonts}
\usepackage{amsthm}

\usepackage{mathrsfs}

% used for TeXing text within eps files
%\usepackage{psfrag}
% need this for including graphics (\includegraphics)
%\usepackage{graphicx}
% for neatly defining theorems and propositions
%
% making logically defined graphics
%%%\usepackage{xypic}

% there are many more packages, add them here as you need them

% define commands here

\newcommand{\sR}[0]{\mathbb{R}}
\newcommand{\sC}[0]{\mathbb{C}}
\newcommand{\sN}[0]{\mathbb{N}}
\newcommand{\sZ}[0]{\mathbb{Z}}

 \usepackage{bbm}
 \newcommand{\Z}{\mathbbmss{Z}}
 \newcommand{\C}{\mathbbmss{C}}
 \newcommand{\R}{\mathbbmss{R}}
 \newcommand{\Q}{\mathbbmss{Q}}



\newcommand*{\norm}[1]{\lVert #1 \rVert}
\newcommand*{\abs}[1]{| #1 |}



\newtheorem{thm}{Theorem}
\newtheorem{defn}{Definition}
\newtheorem{prop}{Proposition}
\newtheorem{lemma}{Lemma}
\newtheorem{cor}{Corollary}
\begin{document}
\subsubsection{Convex sets}
\begin{enumerate}
\item convex set
\item polyconvex set
\item extreme point
\item convex combination
\item \PMlinkname{Carathéodory's theorem}{CaratheodorysTheorem2}
\item Radon's lemma
\item Helly's theorem
\end{enumerate}

\subsubsection{Topological properties of convex sets}
\begin{enumerate}
\item Krein-Milman theorem 
\end{enumerate}

\subsubsection{Convex functions}
\begin{enumerate}
\item convex function
\item extremal value of convex/concave functions
\item logarithmically convex function 
\item Jensen's inequality 
\item subdifferentiable mapping 
\end{enumerate}

\subsubsection{Miscellaneous}
\begin{enumerate}
\item Fenchel's duality theorem
\item saddle function
\item minimax theory
\end{enumerate}
%%%%%
%%%%%
\end{document}
