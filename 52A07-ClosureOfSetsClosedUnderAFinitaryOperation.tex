\documentclass[12pt]{article}
\usepackage{pmmeta}
\pmcanonicalname{ClosureOfSetsClosedUnderAFinitaryOperation}
\pmcreated{2013-03-22 17:03:29}
\pmmodified{2013-03-22 17:03:29}
\pmowner{yark}{2760}
\pmmodifier{yark}{2760}
\pmtitle{closure of sets closed under a finitary operation}
\pmrecord{20}{39349}
\pmprivacy{1}
\pmauthor{yark}{2760}
\pmtype{Theorem}
\pmcomment{trigger rebuild}
\pmclassification{msc}{52A07}
\pmclassification{msc}{57N17}
\pmclassification{msc}{13J99}
\pmclassification{msc}{22A05}
\pmrelated{ClosureOfAVectorSubspaceIsAVectorSubspace}
\pmrelated{ClosureOfAVectorSubspaceIsAVectorSubspace2}
\pmrelated{FreelyGeneratedInductiveSet}

\usepackage{amsthm}

\def\closure{\overline}

\newtheorem*{thm*}{Theorem}
\begin{document}
\PMlinkescapeword{algebraic}
\PMlinkescapeword{closed}
\PMlinkescapeword{closure}
\PMlinkescapephrase{convex subset}
\PMlinkescapeword{inversion}
\PMlinkescapeword{operation}
\PMlinkescapeword{operations}
\PMlinkescapeword{theorem}

In this entry we give a theorem that generalizes such results as
``the \PMlinkname{closure}{Closure} of a subgroup is a subgroup''
and ``the closure of a convex set is convex''.

\section*{Theorem and proof}

Since the theorem involves two different concepts of closure 
--- algebraic and topological --- we must be careful how we phrase it.

\begin{thm*}
Let $X$ be a topological space with a continuous
\PMlinkname{$n$-ary operation}{AlgebraicSystem} $X^n\to X$.
If $A\subseteq X$ is closed under this operation,
then so is $\closure{A}$.
\end{thm*}

{\bf Proof}

Let $\beta$ be the $n$-ary operation,
and suppose that $A$ is closed under this operation,
that is, $\beta(A\times\cdots\times A)\subseteq A$.
From the fact that the 
\PMlinkname{closure of a product is the product of the closures}{ProductTopology},
we have
\[
  \beta(\closure{A}\times\cdots\times\closure{A}) =
  \beta(\closure{A\times\cdots\times A}).
\]
From the
\PMlinkname{characterization of continuity in terms of closure}{TestingForContinuityViaClosureOperation},
we have
\[
  \beta(\closure{A\times\cdots\times A}) \subseteq
  \closure{\beta(A\times\cdots\times A)}.
\]
From the assumption that $\beta(A\times\cdots\times A)\subseteq A$,
we have
\[
  \closure{\beta(A\times\cdots\times A)} \subseteq \closure{A}.
\]
Putting all this together gives
\[
  \beta(\closure{A}\times\cdots\times\closure{A}) \subseteq \closure{A},
\]
as required.

\section*{Examples}

If $H$ is a subgroup of a topological group $G$,
then $H$ is closed under both the group operation
and the operation of inversion,
both of which are continuous,
and therefore by the theorem $\closure{H}$
is also closed under both operations.
Thus the closure of a subgroup of a topological group is also a subgroup.

It similarly follows that the closure of a normal subgroup 
of a topological group is a normal subgroup.
In this case there are additional unary operations to consider:
the maps $x\mapsto g^{-1}xg$ for each $g$ in the group.
But these maps are all continuous, so the theorem again applies.

Note that it does not follow that the closure of a characteristic subgroup
of a topological group is characteristic,
because this would require applying the theorem 
to arbitrary automorphisms of the group,
and these automorphisms need not be continuous.

Straightforward application of the theorem also shows that
the closure of a subring of a topological ring is a subring.
Considering also the unary operations $x\mapsto rx$ for each $r$ in the ring,
we see that the closure of a left ideal of a topological ring is a left ideal.
Similarly, the closure of a right ideal of a topological ring is a right ideal.

We also see that
the closure of a vector subspace of a topological vector space
is a vector subspace.
In this case the operations to consider are vector addition
and for each scalar $\lambda$ the unary operation $x\mapsto\lambda x$.

As a final example, we look at convex sets.
Let $A$ be a convex subset of a real (or complex) topological vector space.
Convexity means that for every $t\in[0,1]$ 
the set is closed under the binary operation $(x,y)\mapsto(1-t)x+ty$.
These binary operations are all continuous,
so the theorem again applies, and we conclude that $\closure{A}$ is convex.

%%%%%
%%%%%
\end{document}
