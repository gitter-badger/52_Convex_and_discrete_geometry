\documentclass[12pt]{article}
\usepackage{pmmeta}
\pmcanonicalname{ProofOfCaratheodorysTheorem}
\pmcreated{2013-03-22 17:50:08}
\pmmodified{2013-03-22 17:50:08}
\pmowner{kshum}{5987}
\pmmodifier{kshum}{5987}
\pmtitle{proof of Carath\'eodory's theorem}
\pmrecord{4}{40306}
\pmprivacy{1}
\pmauthor{kshum}{5987}
\pmtype{Proof}
\pmcomment{trigger rebuild}
\pmclassification{msc}{52A20}

\endmetadata

% this is the default PlanetMath preamble.  as your knowledge
% of TeX increases, you will probably want to edit this, but
% it should be fine as is for beginners.

% almost certainly you want these
\usepackage{amssymb}
\usepackage{amsmath}
\usepackage{amsfonts}

% used for TeXing text within eps files
%\usepackage{psfrag}
% need this for including graphics (\includegraphics)
%\usepackage{graphicx}
% for neatly defining theorems and propositions
%\usepackage{amsthm}
% making logically defined graphics
%%%\usepackage{xypic}

% there are many more packages, add them here as you need them

% define commands here

\begin{document}
The convex hull of $P$ consists precisely of the points that can be written as convex combination of finitely many number points in $P$. Suppose that $p$ is a convex combination of $n$ points in $P$, for some integer $n$,
\[
p = \alpha_1x_1 + \alpha_2 x_2 + \ldots + \alpha_n x_n
\]
where $\alpha_1+\ldots +\alpha_n = 1$ and $x_1, \ldots, x_n \in P$. If $n \leq d+1$, then it is already in the required form.

If $n > d+1$, the $n-1$ points $x_2-x_1$, $x_3-x_1, \ldots, x_n-x_1$ are linearly dependent. Let $\beta_i$, $i=2,3,\ldots, n$, be real numbers, which are not all zero, such that
\[
 \sum_{i=2}^n \beta_i (x_i-x_1) = 0.
\]
So, there are $n$ constants $\gamma_1, \ldots \gamma_n$, not all equal to zero, such that \[\sum_{i=1}^n \gamma_i x_i = 0,\]
and
\[
 \sum_{i=1}^n \gamma_i = 0.
\]

Let $\mathcal{I}$ be a subset of indices defined as
\[\{i\in\{1,2,\ldots, n\}:\, \gamma_i > 0\}.
\]
Since $ \sum_{i=1}^n \gamma_i = 0$, the subset $\mathcal{I}$ is not empty. Define
\[
 a = \max_{i\in I} \alpha_i/\gamma_i.
\]
Then we have
\[
 p = \sum_{i=1}^n (\alpha_i - a\gamma_i) x_i, 
\]
which is a convex combination with at least one zero coefficient. Therefore, we can assume that $p$ can be written as a convex combination of $n-1$ points in $P$, whenever $n > d+1$.

After repeating the above process several times, we can express $p$ as a convex combination of at most $d+1$ points in $P$.
%%%%%
%%%%%
\end{document}
