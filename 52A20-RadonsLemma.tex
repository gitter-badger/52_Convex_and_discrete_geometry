\documentclass[12pt]{article}
\usepackage{pmmeta}
\pmcanonicalname{RadonsLemma}
\pmcreated{2013-03-22 13:56:48}
\pmmodified{2013-03-22 13:56:48}
\pmowner{bbukh}{348}
\pmmodifier{bbukh}{348}
\pmtitle{Radon's lemma}
\pmrecord{8}{34710}
\pmprivacy{1}
\pmauthor{bbukh}{348}
\pmtype{Theorem}
\pmcomment{trigger rebuild}
\pmclassification{msc}{52A20}
\pmclassification{msc}{52C07}
%\pmkeywords{convex hull}

\usepackage{amssymb}
\usepackage{amsmath}
\usepackage{amsfonts}
\usepackage{amsthm}

%%%\usepackage{xypic}

\makeatletter
\@ifundefined{bibname}{}{\renewcommand{\bibname}{References}}
\makeatother
\begin{document}
Every set $A\subset \mathbb{R}^d$ of $d+2$ or more points can be partitioned into two disjoint sets $A_1$ and $A_2$ such that the convex hulls of $A_1$ and $A_2$ intersect.

\begin{proof}
Without loss of generality we assume that the set $A$ consists of exactly $d+2$ points which we number $a_1, a_2,\dotsc, a_{d+2}$. Denote by $a_{i,j}$ the $j$'th component of $i$'th vector, and write the components in a matrix as
\begin{equation*}
M=\begin{bmatrix}
a_{1,1}&a_{2,1}&\dots&a_{d+2,1}\\
a_{1,2}&a_{2,2}&\dots&a_{d+2,2}\\
\vdots& \vdots&\ddots & \vdots\\
a_{1,d}&a_{2,d}&\dots&a_{d+2,d}\\
1&1&\dots&1
\end{bmatrix}.
\end{equation*}
Since $M$ has fewer rows than columns, there is a non-zero column vector $\mathbf{\lambda}$ such that $M \mathbf{\lambda}=0$, which is equivalent to the existence of a solution to the system
%aligned inside equation with a tag does not work with LaTeX2HTML
\begin{align}\label{eqn:sys}
0&=\lambda_1 a_1+\lambda_2 a_2+\dotsb+\lambda_{d+2} a_{d+2}\\
0&=\lambda_1+\lambda_2+\dotsb+\lambda_{d+2}\notag
\end{align}
Let $A_1$ be the set of those $a_i$ for which $\lambda_i$ is positive, and $A_2$ be the rest. Set $s$ to be the sum of positive $\lambda_i$'s. Then by the system~\eqref{eqn:sys} above 
\begin{equation*}
\frac{1}{s}\sum_{a_i\in A_1} \lambda_i a_i=\frac{1}{s}\sum_{a_i\in A_2} (-\lambda_i) a_i
\end{equation*}
is a point of intersection of convex hulls of $A_1$ and $A_2$.
\end{proof}
%%%%%
%%%%%
\end{document}
