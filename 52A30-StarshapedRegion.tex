\documentclass[12pt]{article}
\usepackage{pmmeta}
\pmcanonicalname{StarshapedRegion}
\pmcreated{2013-03-22 13:34:13}
\pmmodified{2013-03-22 13:34:13}
\pmowner{matte}{1858}
\pmmodifier{matte}{1858}
\pmtitle{star-shaped region}
\pmrecord{10}{34182}
\pmprivacy{1}
\pmauthor{matte}{1858}
\pmtype{Definition}
\pmcomment{trigger rebuild}
\pmclassification{msc}{52A30}
\pmclassification{msc}{32F99}
\pmdefines{star-shaped}

\endmetadata

% this is the default PlanetMath preamble.  as your knowledge
% of TeX increases, you will probably want to edit this, but
% it should be fine as is for beginners.

% almost certainly you want these
\usepackage{amssymb}
\usepackage{amsmath}
\usepackage{amsfonts}

% used for TeXing text within eps files
%\usepackage{psfrag}
% need this for including graphics (\includegraphics)
%\usepackage{graphicx}
% for neatly defining theorems and propositions
%\usepackage{amsthm}
% making logically defined graphics
%%%\usepackage{xypic}

% there are many more packages, add them here as you need them

% define commands here

\newcommand{\sR}[0]{\mathbb{R}}
\newcommand{\sC}[0]{\mathbb{C}}
\newcommand{\sN}[0]{\mathbb{N}}
\newcommand{\sZ}[0]{\mathbb{Z}}
\begin{document}
{\bf Definition}
A subset $U$ of a real (or possibly complex) vector space is called 
\emph{star-shaped} if there is a point $p\in U$ such that the line segment
$\overline{pq}$ is contained in $U$ for all $q\in U$. (Here, $\overline{pq} = \{ tp + (1-t)q\, | \, t\in[0,1] \}$.)  We then say that $U$ is star-shaped with respect to $p$.

In other \PMlinkescapetext{words}, a region $U$ is star-shaped if there is a point $p\in U$ such that $U$ can be ``collapsed'' or ``contracted'' \PMlinkescapetext{onto} $p$.

\subsubsection{Examples}
\begin{enumerate}
\item In $\sR^n$, any vector subspace is star-shaped. Also, the unit cube and 
unit ball are star-shaped, but the unit sphere is not. 
\item A subset $U$ of a vector space is star-shaped with respect to all of its points if and only if $U$ is convex. 
\end{enumerate}
%%%%%
%%%%%
\end{document}
