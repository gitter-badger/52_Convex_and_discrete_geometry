\documentclass[12pt]{article}
\usepackage{pmmeta}
\pmcanonicalname{HellysTheorem}
\pmcreated{2013-03-22 13:57:38}
\pmmodified{2013-03-22 13:57:38}
\pmowner{bbukh}{348}
\pmmodifier{bbukh}{348}
\pmtitle{Helly's theorem}
\pmrecord{6}{34728}
\pmprivacy{1}
\pmauthor{bbukh}{348}
\pmtype{Theorem}
\pmcomment{trigger rebuild}
\pmclassification{msc}{52A35}

\usepackage{amssymb}
\usepackage{amsmath}
\usepackage{amsfonts}

\usepackage{amsthm}
\DeclareMathOperator{\conv}{conv}

\makeatletter
\@ifundefined{bibname}{}{\renewcommand{\bibname}{References}}
\makeatother
\begin{document}
Suppose $A_1,\dotsc,A_m\subset \mathbb{R}^d$ is a family of convex sets, and every $d+1$ of them have a non-empty intersection. Then $\bigcap_{i=1}^m A_i$ is non-empty.
\begin{proof}
The proof is by induction on $m$. If $m=d+1$, then the statement is vacuous. Suppose the statement is true if $m$ is replaced by $m-1$. The sets $B_j=\bigcap_{i\neq j} A_i$ are non-empty by inductive hypothesis. Pick a point $p_j$ from each of $B_j$. By Radon's lemma, there is a partition of $p$'s into two sets $P_1$ and $P_2$ such that $I=(\conv P_1)\cap(\conv P_2)\neq \emptyset$. For every $A_j$ either every point in $P_1$ belongs to $A_j$ or every point in $P_2$ belongs to $A_j$. Hence $I\subseteq A_j$ for every $j$.
\end{proof}
%%%%%
%%%%%
\end{document}
