\documentclass[12pt]{article}
\usepackage{pmmeta}
\pmcanonicalname{ConvexFunction}
\pmcreated{2013-03-22 11:46:26}
\pmmodified{2013-03-22 11:46:26}
\pmowner{matte}{1858}
\pmmodifier{matte}{1858}
\pmtitle{convex function}
\pmrecord{28}{30231}
\pmprivacy{1}
\pmauthor{matte}{1858}
\pmtype{Definition}
\pmcomment{trigger rebuild}
\pmclassification{msc}{52A41}
\pmclassification{msc}{26A51}
\pmclassification{msc}{26B25}
\pmclassification{msc}{55Q05}
\pmclassification{msc}{18G30}
\pmclassification{msc}{18B40}
\pmclassification{msc}{20J05}
\pmclassification{msc}{20E07}
\pmclassification{msc}{18-01}
\pmclassification{msc}{20L05}
\pmrelated{JensensInequality}
\pmrelated{LogarithmicallyConvexFunction}
\pmdefines{concave function}
\pmdefines{strictly convex function}
\pmdefines{strictly concave function}
\pmdefines{strictly convex}
\pmdefines{strictly concave}
\pmdefines{epigraph}
\pmdefines{effective domain}
\pmdefines{concave}

\endmetadata

% this is the default PlanetMath preamble.  as your knowledge
% of TeX increases, you will probably want to edit this, but
% it should be fine as is for beginners.

% almost certainly you want these
\usepackage{amssymb}
\usepackage{amsmath}
\usepackage{amsfonts}

% used for TeXing text within eps files
%\usepackage{psfrag}
% need this for including graphics (\includegraphics)
%\usepackage{graphicx}
% for neatly defining theorems and propositions
%\usepackage{amsthm}
% making logically defined graphics
%%%%%\usepackage{xypic}

% there are many more packages, add them here as you need them

% define commands here

\newcommand{\sR}[0]{\mathbb{R}}
\newcommand{\sC}[0]{\mathbb{C}}
\newcommand{\sN}[0]{\mathbb{N}}
\newcommand{\sZ}[0]{\mathbb{Z}}
\begin{document}
{\bf Definition} Suppose $\Omega$ is a convex set in a vector space over $\sR$
(or $\sC$), and suppose $f$ is a function $f:\Omega\to \sR$.
If for any $x,y\in \Omega$, $x\neq y$ and any $\lambda \in (0,1)$, we have
 $$f\Big( \lambda x + (1-\lambda)y\Big)\leq \lambda f(x)+(1-\lambda)f(y),$$
we say that $f$ is a \emph{convex function}.
If for any $x,y\in \Omega$ and any $\lambda \in (0,1)$, we have
 $$f\Big( \lambda x + (1-\lambda)y\Big)\geq \lambda f(x)+(1-\lambda)f(y),$$
we say that $f$ is a \emph{concave function}. If either of the  inequalities
are strict, then we say that $f$ is a \emph{strictly convex function},
or a \emph{strictly concave function}, respectively.

\subsubsection*{Properties}
 \begin{itemize}
\item A function $f$ is a (strictly) convex function if and only if $-f$ is
a (strictly) concave function. For this reason, most of the below discussion only 
focuses on convex functions. Analogous result holds for concave functions. 
 \item On $\sR$, a continuous function is convex 
if and only if for all $x,y\in \sR$, we have
 $$f\left(\frac{x+y}{2}\right)\le\frac{f(x)+f(y)}{2}.$$
\item  On $\sR$, a once differentiable function is convex if and only if $f'$
is monotone increasing. 
 \item Suppose $f$ is twice continuously differentiable function on $\sR$.
Then $f$ is convex if and only if $f'' \ge 0$. 
If $f''>0$, then $f$ is strictly convex.
\item A local minimum of a convex function is a global minimum.
See \PMlinkname{this page}{ExtremalValueOfConvexconcaveFunctions}.
 \end{itemize}


\subsubsection*{Examples}
\begin{itemize}
\item $e^x$,$e^{-x}$, and $x^2$ are convex functions on $\sR$. Also, $x^4$ is strictly
convex, but $12x^2$ vanishes at $x=0$. 
\item A \PMlinkname{norm}{NormedVectorSpace} is a convex function.
\end{itemize}

\subsubsection*{Remark.}  
We may generalize the above definition of a convex function to an that of an extended real-valued function whose domain is not necessarily a convex set.  First, we define what an \emph{epigraph} of a function is.

Let $\Omega$ be a subset of a vector space over the reals, and $f$ an extended real-valued function defined on $\Omega$.  The \emph{epigraph} of $f$, denoted by $\operatorname{epi}(f)$, is the set 
$$\lbrace (x,r)\mid x\in \Omega,\mbox{ } r\ge f(x)\rbrace.$$
An extended real-valued function $f$ defined on a subset $\Omega$ of a vector space $V$ over the reals is said to be \emph{convex} if its \emph{epigraph} is a convex subset of $V\times\mathbb{R}$.  With this definition, the domain $\Omega$ of $f$ need not be convex.  However, its subset $\lbrace x\in \Omega\mid f(x) < \infty\rbrace$, called the \emph{effective domain} and denoted by $\operatorname{eff.dom}(f)$, is convex.  To see this, suppose $x,y\in \operatorname{eff.dom}(f)$ and $z=\lambda x+(1-\lambda) y$ with $0\le \lambda \le 0$.  Then $ (z,\overline{z})=\lambda(x,f(x))+(1-\lambda)(y,f(y))\in \operatorname{epi}(f)$, where $\overline{z}=\lambda f(x)+(1-\lambda)f(y)$, since $\operatorname{epi}(f)$ is convex by definition.  Therefore, $z\in \operatorname{dom}(f)$.  In fact, $f(z)\le \overline{z}=\lambda f(x)+(1-\lambda)f(y)<\infty$, which implies that $z\in \operatorname{eff.dom}(f)$.
%%%%%
%%%%%
%%%%%
%%%%%
\end{document}
