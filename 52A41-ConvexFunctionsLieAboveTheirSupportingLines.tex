\documentclass[12pt]{article}
\usepackage{pmmeta}
\pmcanonicalname{ConvexFunctionsLieAboveTheirSupportingLines}
\pmcreated{2013-03-22 16:59:20}
\pmmodified{2013-03-22 16:59:20}
\pmowner{Andrea Ambrosio}{7332}
\pmmodifier{Andrea Ambrosio}{7332}
\pmtitle{convex functions lie above their supporting lines}
\pmrecord{5}{39268}
\pmprivacy{1}
\pmauthor{Andrea Ambrosio}{7332}
\pmtype{Result}
\pmcomment{trigger rebuild}
\pmclassification{msc}{52A41}
\pmclassification{msc}{26A51}
\pmclassification{msc}{26B25}

\endmetadata

% this is the default PlanetMath preamble.  as your knowledge
% of TeX increases, you will probably want to edit this, but
% it should be fine as is for beginners.

% almost certainly you want these
\usepackage{amssymb}
\usepackage{amsmath}
\usepackage{amsfonts}

% used for TeXing text within eps files
%\usepackage{psfrag}
% need this for including graphics (\includegraphics)
%\usepackage{graphicx}
% for neatly defining theorems and propositions
\usepackage{amsthm}
% making logically defined graphics
%%%\usepackage{xypic}

% there are many more packages, add them here as you need them

% define commands here

\begin{document}
Let $f:\mathbf{R}\rightarrow \mathbf{R}$ be a convex, twice differentiable
function on $[a,b]$. Then $f(x)$ lies above its supporting lines, i.e. it's
greater than any tangent line in $[a,b]$.

\begin{proof}:

Let $r(x)=f\left( x_{0}\right) +f^{\prime }\left( x_{0}\right) \left(
x-x_{0}\right) $ be the tangent of $f(x)$ in $x=x_{0}\in \lbrack a,b].$

By Taylor theorem, with remainder in Lagrange form, one has, for any $x\in
\lbrack a,b]$:
\[
f\left( x\right) =f\left( x_{0}\right) +f^{\prime }\left( x_{0}\right)
\left( x-x_{0}\right) +\frac{1}{2}f^{^{\prime \prime }}\left( \xi \left(
x\right) \right) \left( x-x_{0}\right) ^{2}
\]
with $\xi \left( x\right) \in \lbrack a,b]$. Then
\[
f\left( x\right) -r(x)=\frac{1}{2}f^{^{\prime \prime }}\left( \xi \left(
x\right) \right) \left( x-x_{0}\right) ^{2}\geq 0
\]
since $f^{^{\prime \prime }}\left( \xi \left( x\right) \right) \geq 0$ by
convexity.
\end{proof}
%%%%%
%%%%%
\end{document}
