\documentclass[12pt]{article}
\usepackage{pmmeta}
\pmcanonicalname{ExtremePoint}
\pmcreated{2013-03-22 14:24:55}
\pmmodified{2013-03-22 14:24:55}
\pmowner{jirka}{4157}
\pmmodifier{jirka}{4157}
\pmtitle{extreme point}
\pmrecord{7}{35920}
\pmprivacy{1}
\pmauthor{jirka}{4157}
\pmtype{Definition}
\pmcomment{trigger rebuild}
\pmclassification{msc}{52A99}
\pmrelated{FaceOfAConvexSet}
\pmrelated{ExposedPointsAreDenseInTheExtremePoints}

\endmetadata

% this is the default PlanetMath preamble.  as your knowledge
% of TeX increases, you will probably want to edit this, but
% it should be fine as is for beginners.

% almost certainly you want these
\usepackage{amssymb}
\usepackage{amsmath}
\usepackage{amsfonts}

% used for TeXing text within eps files
%\usepackage{psfrag}
% need this for including graphics (\includegraphics)
%\usepackage{graphicx}
% for neatly defining theorems and propositions
\usepackage{amsthm}
% making logically defined graphics
%%%\usepackage{xypic}

% there are many more packages, add them here as you need them

% define commands here
\theoremstyle{theorem}
\newtheorem*{thm}{Theorem}
\newtheorem*{lemma}{Lemma}
\newtheorem*{conj}{Conjecture}
\newtheorem*{cor}{Corollary}
\newtheorem*{example}{Example}
\theoremstyle{definition}
\newtheorem*{defn}{Definition}
\begin{document}
\begin{defn}
Let $C$ be a convex subset of a vector space $X$.  A point $x \in C$ is
called an {\em extreme point} if it is not an interior point of any line segment
in $C$.  That is $x$ is extreme if and only if whenever $x = ty +(1-t)z$, $t \in (0,1)$, $z \not= y$, implies either $y \notin C$ or $z \notin C$.
\end{defn}

For example the set $[0,1] \in {\mathbb{R}}$ is a convex set and $0$ and $1$ are the extreme points.

\begin{thebibliography}{9}
\bibitem{royden}
H.\@ L.\@ Royden. \emph{\PMlinkescapetext{Real Analysis}}. Prentice-Hall, Englewood Cliffs, New Jersey, 1988
\end{thebibliography}
%%%%%
%%%%%
\end{document}
