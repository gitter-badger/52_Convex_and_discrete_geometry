\documentclass[12pt]{article}
\usepackage{pmmeta}
\pmcanonicalname{ExtremeSubsetOfConvexSet}
\pmcreated{2013-03-22 15:24:43}
\pmmodified{2013-03-22 15:24:43}
\pmowner{georgiosl}{7242}
\pmmodifier{georgiosl}{7242}
\pmtitle{extreme subset of convex set}
\pmrecord{7}{37252}
\pmprivacy{1}
\pmauthor{georgiosl}{7242}
\pmtype{Definition}
\pmcomment{trigger rebuild}
\pmclassification{msc}{52A99}
\pmrelated{ConvexSet}

% this is the default PlanetMath preamble.  as your knowledge
% of TeX increases, you will probably want to edit this, but
% it should be fine as is for beginners.

% almost certainly you want these
\usepackage{amssymb}
\usepackage{amsmath}
\usepackage{amsfonts}

% used for TeXing text within eps files
%\usepackage{psfrag}
% need this for including graphics (\includegraphics)
%\usepackage{graphicx}
% for neatly defining theorems and propositions
%\usepackage{amsthm}
% making logically defined graphics
%%%\usepackage{xypic}

% there are many more packages, add them here as you need them

% define commands here
\begin{document}
Let $K$ a non-empty closed \PMlinkname{convex subset}{ConvexSet} of a normed vector space. A set $A\subseteq K$ is called an \emph{extreme subset} of $K$
if $A$ is closed, convex and satisfies the condition $\colon$ for any $x,y \in K$ and $tx+(1-t)y \in A,
t\in (0,1)$ then $x, y \in A$.

For example let $K=[0,1]\times[0,1]$ then $K$, sides of $K$, included the endpoints, and $\{(1,1),(0,1),(1,0),(0,0)\}$ are extreme subsets of $K$.
%%%%%
%%%%%
\end{document}
