\documentclass[12pt]{article}
\usepackage{pmmeta}
\pmcanonicalname{FaceOfAConvexSet}
\pmcreated{2013-03-22 16:23:08}
\pmmodified{2013-03-22 16:23:08}
\pmowner{CWoo}{3771}
\pmmodifier{CWoo}{3771}
\pmtitle{face of a convex set}
\pmrecord{12}{38530}
\pmprivacy{1}
\pmauthor{CWoo}{3771}
\pmtype{Definition}
\pmcomment{trigger rebuild}
\pmclassification{msc}{52A99}
\pmrelated{ExtremePoint}
\pmdefines{face}
\pmdefines{proper face}
\pmdefines{extreme point}
\pmdefines{improper face}

\usepackage{amssymb,amscd}
\usepackage{amsmath}
\usepackage{amsfonts}

% used for TeXing text within eps files
%\usepackage{psfrag}
% need this for including graphics (\includegraphics)
%\usepackage{graphicx}
% for neatly defining theorems and propositions
%\usepackage{amsthm}
% making logically defined graphics
%%\usepackage{xypic}
\usepackage{pst-plot}
\usepackage{psfrag}

% define commands here

\begin{document}
Let $C$ be a convex set in $\mathbb{R}^n$ (or any topological vector
space).  A \emph{face} of $C$ is a subset $F$ of $C$ such that
\begin{enumerate}
\item $F$ is convex, and 
\item given any line segment $L\subseteq C$, if
$\operatorname{ri}(L)\cap F\ne \varnothing$, then $L\subseteq F$.
\end{enumerate}
Here, $\operatorname{ri}(L)$ denotes the relative interior of $L$
(open segment of $L$).

A zero-dimensional face of a convex set $C$ is called an \emph{extreme
point} of $C$.

This definition formalizes the notion of a face of a convex polygon or
a convex polytope and generalizes it to an arbitrary convex set.  For
example, any point on the boundary of a closed unit disk in
$\mathbb{R}^2$ is its face (and an extreme point).

Observe that the empty set and $C$ itself are faces of $C$.  These
faces are sometimes called \emph{improper faces}, while other faces
are called \emph{proper faces}.

\textbf{Remarks}.
Let $C$ be a convex set.
\begin{itemize}
\item The intersection of two faces of $C$ is a face of $C$.
\item A face of a face of $C$ is a face of $C$.
\item Any proper face of $C$ lies on its relative boundary,
$\operatorname{rbd}(C)$.
\item The set $\operatorname{Part}(C)$ of all relative interiors of the faces of $C$ partitions $C$.
\item If $C$ is compact, then $C$ is the convex hull of its extreme points.
\item The set $F(C)$ of faces of a convex set $C$ forms a lattice, where the meet is the intersection: $F_1 \wedge F_2 := F_1\cap F_2$; the join of $F_1,F_2$ is the smallest face $F\in F(C)$ containing both $F_1$ and $F_2$.  This lattice is bounded lattice (by $\varnothing$ and $C$).  And it is not hard to see that $F(C)$ is a complete lattice.
\item
However, in general, $F(C)$ is not a modular lattice.  As a counterexample, consider the unit square $[0,1]\times [0,1]$ and faces $a=(0,0)$, $b=\lbrace (0,y)\mid y\in [0,1]\rbrace$, and $c=(1,1)$.  We have $a\le b$.  However, $a\vee (b\wedge c)=(0,0)\vee \varnothing=(0,0)$, whereas $(a\vee b)\wedge c = b\wedge \varnothing=\varnothing$.
\item Nevertheless, $F(C)$ is a complemented lattice.  Pick any face $F\in F(C)$.  If $F=C$, then $\varnothing$ is a complement of $F$.  Otherwise, form $\operatorname{Part}(C)$ and $\operatorname{Part}(F)$, the partitions of $C$ and $F$ into disjoint unions of the relative interiors of their corresponding faces.  Clearly $\operatorname{Part}(F)\subset \operatorname{Part}(C)$ strictly.  Now, it is possible to find an extreme point $p$ such that $\lbrace p\rbrace\in \operatorname{Part}(C)-\operatorname{Part}(F)$.  Otherwise, all extreme points lie in $\operatorname{Part}(F)$, which leads to 
$$\operatorname{Part}(F) = \operatorname{Part}(\mbox{convex hull of extreme points of }C)=\operatorname{Part}(C),$$ a contradiction.  Finally, let $G$ be the convex hull of extreme points of $C$ not contained in $\operatorname{Part}(F)$.  We assert that $G$ is a complement of $F$.  If $x\in G\cap F$, then $G\cap F$ is a proper face of $G$ and of $F$, hence its extreme points are also extreme points of $G$, and of $F$, which is impossible by the construction of $G$.  Therefore $F\cap G=\varnothing$.  Next, note that the union of extreme points of $G$ and of $F$ is the collection of all extreme points of $C$, this is again the result of the construction of $G$, so any $y\in C$ is in the join of all its extreme points, which is equal to the join of $F$ and $G$ (since join is universally associative).
\item 
Additionally, in $F(C)$, zero-dimensional faces are compact elements, and compact elements are faces with finitely many extreme points.  The unit disk $D$ is not compact in $F(D)$.  Since every face is the convex hull (join) of all extreme points it contains, $F(C)$ is an algebraic lattice.
\end{itemize}

\begin{thebibliography}{9}
\bibitem{Rockafellar} R.T. Rockafellar, \emph{Convex Analysis}, Princeton University Press, 1996.
\end{thebibliography}
%%%%%
%%%%%
\end{document}
