\documentclass[12pt]{article}
\usepackage{pmmeta}
\pmcanonicalname{FaceOfAConvexSetAlternativeDefinitionOf}
\pmcreated{2013-03-22 17:02:02}
\pmmodified{2013-03-22 17:02:02}
\pmowner{mps}{409}
\pmmodifier{mps}{409}
\pmtitle{face of a convex set, alternative definition of}
\pmrecord{4}{39320}
\pmprivacy{1}
\pmauthor{mps}{409}
\pmtype{Definition}
\pmcomment{trigger rebuild}
\pmclassification{msc}{52A99}
\pmsynonym{face}{FaceOfAConvexSetAlternativeDefinitionOf}
\pmdefines{supporting hyperplane}

% this is the default PlanetMath preamble.  as your knowledge
% of TeX increases, you will probably want to edit this, but
% it should be fine as is for beginners.

% almost certainly you want these
\usepackage{amssymb}
\usepackage{amsmath}
\usepackage{amsfonts}

% used for TeXing text within eps files
%\usepackage{psfrag}
% need this for including graphics (\includegraphics)
%\usepackage{graphicx}
% for neatly defining theorems and propositions
%\usepackage{amsthm}
% making logically defined graphics
%%%\usepackage{xypic}

% there are many more packages, add them here as you need them

% define commands here

\begin{document}
The following definition of a face of a convex set in a real vector
space is sometimes useful.

Let $C$ be a convex subset of $\mathbb{R}^n$.  Before we define faces,
we introduce oriented hyperplanes and supporting hyperplanes.

Given any vectors $n$ and $p$ in $\mathbb{R}^n$, define the hyperplane
$H(n,p)$ by
\[
  H(n,p) = \{ x \in \mathbb{R}^n \colon n \cdot (x - p) = 0 \};
\]
note that this is the degenerate hyperplane $\mathbb{R}^n$ if $n=0$.
As long as $H(n,p)$ is nondegenerate, its removal disconnects
$\mathbb{R}^n$.  The \emph{upper halfspace} of $\mathbb{R}^n$ determined by
$H(n,p)$ is
\[
   H(n,p)^+ = \{ x \in \mathbb{R}^n \colon n \cdot (x - p) \ge 0 \}.
\]
A hyperplane $H(n,p)$ is a \emph{supporting hyperplane} for
$C$ if its upper halfspace contains $C$, that is, if $C\subset H(n.p)^+$.

Using this terminology, we can define a \emph{face} of a convex set
$C$ to be the intersection of $C$ with a supporting hyperplane of $C$.
Notice that we still get the empty set and $C$ as improper faces of $C$.

\textbf{Remarks.}  Let $C$ be a convex set.
\begin{itemize}
\item
If $F_1 = C\cap H(n_1,p_1)$ and $F_2 = C\cap H(n_2,p_2)$ are faces
of $C$ intersecting in a point $p$, then $H(n_1+n_2,p)$ is a
supporting hyperplane of $C$, and $F_1\cap F_2 = C\cap H(n_1+n_2,p)$.
This shows that the faces of $C$ form a meet-semilattice.

\item
Since each proper face lies on the base of the upper halfspace of some
supporting hyperplane, each such face must lie on the relative
boundary of $C$.
\end{itemize}

\PMlinkescapeword{base}
%%%%%
%%%%%
\end{document}
