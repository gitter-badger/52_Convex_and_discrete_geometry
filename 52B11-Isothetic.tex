\documentclass[12pt]{article}
\usepackage{pmmeta}
\pmcanonicalname{Isothetic}
\pmcreated{2013-03-22 15:54:33}
\pmmodified{2013-03-22 15:54:33}
\pmowner{lha}{3057}
\pmmodifier{lha}{3057}
\pmtitle{isothetic}
\pmrecord{8}{37912}
\pmprivacy{1}
\pmauthor{lha}{3057}
\pmtype{Definition}
\pmcomment{trigger rebuild}
\pmclassification{msc}{52B11}

\endmetadata

% this is the default PlanetMath preamble.  as your knowledge
% of TeX increases, you will probably want to edit this, but
% it should be fine as is for beginners.

% almost certainly you want these
\usepackage{amssymb}
\usepackage{amsmath}
\usepackage{amsfonts}

% used for TeXing text within eps files
%\usepackage{psfrag}
% need this for including graphics (\includegraphics)
%\usepackage{graphicx}
% for neatly defining theorems and propositions
%\usepackage{amsthm}
% making logically defined graphics
%%%\usepackage{xypic}

% there are many more packages, add them here as you need them

% define commands here
\begin{document}
A curve is {\em isothetic} if it consists entirely of lines parallel to one of the coordinate axes in a given rectilinear coordinate system.  A polygon or polyhedron is isothetic if all of its edges are parallel to one of the coordinate axes.

An example of an isothetic polygon is the rectangle $\{(x,y) : x_1 \le x \le x_2,\; y_1 \le y \le y_2\}$ for some $x_1$, $x_2$, $y_1$, $y_2$.  Examples of non-isothetic shapes are the tilted square $\{(x,y,z) : |x| + |y| = 1 \}$ and the bipyramid $\{(x,y,z) : |x|+|y|+|z|=1 \}$.

(This entry is here because I couldn't find a definition of isothetic on the web.  If you know anything interesting about isothetic shapes, please adopt this entry!)
%%%%%
%%%%%
\end{document}
