\documentclass[12pt]{article}
\usepackage{pmmeta}
\pmcanonicalname{Fvector}
\pmcreated{2013-03-22 16:59:10}
\pmmodified{2013-03-22 16:59:10}
\pmowner{mps}{409}
\pmmodifier{mps}{409}
\pmtitle{f-vector}
\pmrecord{5}{39264}
\pmprivacy{1}
\pmauthor{mps}{409}
\pmtype{Definition}
\pmcomment{trigger rebuild}
\pmclassification{msc}{52B40}
\pmsynonym{$f$-vector}{Fvector}
\pmdefines{flag f-vector}
\pmdefines{flag $f$-vector}

% this is the default PlanetMath preamble.  as your knowledge
% of TeX increases, you will probably want to edit this, but
% it should be fine as is for beginners.

% almost certainly you want these
\usepackage{amssymb}
\usepackage{amsmath}
\usepackage{amsfonts}

% used for TeXing text within eps files
%\usepackage{psfrag}
% need this for including graphics (\includegraphics)
%\usepackage{graphicx}
% for neatly defining theorems and propositions
%\usepackage{amsthm}
% making logically defined graphics
%%%\usepackage{xypic}

% there are many more packages, add them here as you need them

% define commands here
\newcommand{\fm}[1]{{\it #1}}
\begin{document}
Let \fm{P} be a polytope of dimension \fm{d}.  The \emph{f-vector} of
\fm{P} is the finite integer sequence $(f_0, \dots, f_{d-i})$, where
the component in position \fm{i} is the number of \fm{i}-dimensional
faces of \fm{P}.  For some purposes it is convenient to view the empty
face and the polytope itself as improper faces, so $f_{-1} = f_d = 1$.

For example, a cube has 8 vertices, 12 edges, and 6 faces, so its
f-vector is (8, 12, 6).

The entries in the f-vector of a convex polytope satisfy the 
Euler--Poincar\'e--Schl\"afli formula:
\begin{equation*}
\sum_{-1\le i\le d}(-1)^i f_i=0.
\end{equation*}
Consequently, the face lattice of a polytope is Eulerian. For any
graded poset with maximum and minimum elements there is an extension
of the f-vector called the \emph{flag f-vector}.  For any subset
\fm{S} of $\{0,1,\dots,d - 1\}$, the $f_S$ entry of the flag
f-vector of \fm{P} is the number of chains of faces in
$\mathcal{L}(P)$ with dimensions coming only from \fm{S}.

The flag f-vector of a three-dimensional cube is given in the
following table.  For simplicity we drop braces and commas.
\begin{center}
\begin{tabular}{cl}
\fm{S} & $f_S$ \\
\hline
$\emptyset$ & 1  \\
0           & 8  \\
1           & 12 \\
2           & 6  \\
01          & $8\cdot 3 = 24$ \\
02          & $8\cdot 3 = 24$ \\
12          & $12\cdot 2 = 24$ \\
012         & $8\cdot 3\cdot 2 = 48$
\end{tabular}
\end{center}
For example, $f_{\{1,2\}}=24$ because each of the 12 edges
meets exactly two faces.

Although the flag f-vector of a \fm{d}-polytope has $2^d$ entries,
most of them are redundant, as they satisfy a collection of identities
generalizing the Euler--Poincar\'e--Schl\"afli formula and called the
generalized Dehn-Sommerville relations. Interestingly, the number of
nonredundant entries in the flag $f$-vector of a \fm{d}-polytope is
one less than the Fibonacci number $F_{d-1}$.

\begin{thebibliography}{3}
\bibitem{cite:BB}
Bayer, M. and L. Billera, \emph{Generalized Dehn-Sommerville relations for
polytopes, spheres and Eulerian partially ordered sets}, Invent. Math. 79
(1985), no. 1, 143--157.
\bibitem{cite:BK}
Bayer, M. and A. Klapper, \emph{A new index for polytopes}, Discrete Comput.
Geom. 6
(1991), no. 1, 33--47.
\bibitem{cite:Z}
Ziegler, G., \emph{Lectures on polytopes}, Springer-Verlag, 1997.
\end{thebibliography}

\PMlinkescapeword{flag}
%%%%%
%%%%%
\end{document}
