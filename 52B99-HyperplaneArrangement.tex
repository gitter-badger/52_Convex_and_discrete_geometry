\documentclass[12pt]{article}
\usepackage{pmmeta}
\pmcanonicalname{HyperplaneArrangement}
\pmcreated{2013-03-22 15:47:55}
\pmmodified{2013-03-22 15:47:55}
\pmowner{mps}{409}
\pmmodifier{mps}{409}
\pmtitle{hyperplane arrangement}
\pmrecord{6}{37760}
\pmprivacy{1}
\pmauthor{mps}{409}
\pmtype{Definition}
\pmcomment{trigger rebuild}
\pmclassification{msc}{52B99}
\pmclassification{msc}{52C35}
\pmsynonym{subspace arrangement}{HyperplaneArrangement}
\pmsynonym{central arrangement}{HyperplaneArrangement}
%\pmkeywords{arrangement}
%\pmkeywords{partition}
%\pmkeywords{fundamental group}
\pmrelated{Zonotope}
\pmdefines{Grassmannian}

% this is the default PlanetMath preamble.  as your knowledge
% of TeX increases, you will probably want to edit this, but
% it should be fine as is for beginners.

% almost certainly you want these
\usepackage{amssymb}
\usepackage{amsmath}
\usepackage{amsfonts}

% used for TeXing text within eps files
%\usepackage{psfrag}
% need this for including graphics (\includegraphics)
%\usepackage{graphicx}
% for neatly defining theorems and propositions
\usepackage{amsthm}
% making logically defined graphics
%%%\usepackage{xypic}

% there are many more packages, add them here as you need them

% define commands here
\theoremstyle{plain}
\newtheorem{example}{Example}
\DeclareMathOperator{\Gr}{Gr}
\begin{document}
\PMlinkescapeword{between}
\PMlinkescapeword{dimension}
\PMlinkescapeword{equivalent}
Let $V$ be a vector space over a field $\mathbb{K}$.  A 
{\em hyperplane arrangment} in $V$ is a family
$\mathcal{A}=\{\mathcal{H}_i\}_{i\in I}$
of affine hyperplanes in $V$.  If all of the hyperplanes
pass through $0$, $\mathcal{A}$ is called {\em central};
otherwise, it is {\em affine}.  More generally, a 
{\em subspace arrangement} is a family of affine subspaces of $V$. 
The same distinction between central and affine subspace arrangement
holds.

\begin{example}
Let $V=\mathbb{K}^n$.  Then the family
\[
\mathbb{K}P^n=\{S\subset V\mid\dim_{\mathbb{K}}(S)=1\}
\]
of $1$-dimensional subspaces of $V$ is a central subspace
arrangement, the projective space of dimension $n$ over 
$\mathbb{K}$.
\end{example}

Instead of considering all lines through a vector space, 
we could consider all $k$-dimensional subspaces
of the space.

\begin{example}
Again let $V=\mathbb{K}^n$, and suppose $0\le k\le n$.  Then the
family
\[
\Gr(V,k)=\{S\subset V\mid\dim_{\mathbb{K}}(S)=k\}
\]
of $k$-dimensional subspaces of $V$ is a central subspace
arrangement, the Grassmannian.  Observe that 
$\mathbb{K}P^n=\Gr(\mathbb{K}^n,1)$.
\end{example}

If $V$ is a topological vector space and $\mathcal{A}$ is a
hyperplane arrangement, then it makes sense to ask for the
fundamental group of the complement 
$V\setminus\bigcup_{\mathcal{H}\in\mathcal{A}}\mathcal{H}$.

\begin{example}
If $\mathcal{A}$ is a finite hyperplane arrangement over
$V=\mathcal{R}^n$, then the arrangement 
\PMlinkname{partitions}{Partition} $V$
into a finite number of contractible cells.  By selecting
a point in each cell and taking the convex hull of the result,
we obtain a polytope combinatorially equivalent to the 
zonotope dual to the arrangement.  Since the question of
the fundamental group here is not interesting, we could also
use the embedding $\mathbb{R}^n\hookrightarrow\mathbb{C}^n$
to complexify $\mathcal{A}$.  In this case the
complement 
$\mathbb{C}^n\setminus\bigcup_{\mathcal{H}\in\mathcal{A}}\mathcal{H}$
usually has nontrivial fundamental group.
\end{example}

\begin{thebibliography}{9}
\bibitem{KR}
Klain, D.\ A., and G.-C.\ Rota, {\em, Introduction to geometric probability},
Cambridge University Press, 1997.
\bibitem{OT}
Orlik, P., and H.\ Terao, {\em Arrangements of hyperplanes}, 
Springer-Verlag, 1992.
\end{thebibliography}
%%%%%
%%%%%
\end{document}
