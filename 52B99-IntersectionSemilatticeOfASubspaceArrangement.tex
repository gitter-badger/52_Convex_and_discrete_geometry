\documentclass[12pt]{article}
\usepackage{pmmeta}
\pmcanonicalname{IntersectionSemilatticeOfASubspaceArrangement}
\pmcreated{2013-03-22 15:47:58}
\pmmodified{2013-03-22 15:47:58}
\pmowner{CWoo}{3771}
\pmmodifier{CWoo}{3771}
\pmtitle{intersection semilattice of a subspace arrangement}
\pmrecord{8}{37761}
\pmprivacy{1}
\pmauthor{CWoo}{3771}
\pmtype{Definition}
\pmcomment{trigger rebuild}
\pmclassification{msc}{52B99}
\pmclassification{msc}{52C35}
\pmsynonym{intersection lattice}{IntersectionSemilatticeOfASubspaceArrangement}
\pmsynonym{intersection semilattice}{IntersectionSemilatticeOfASubspaceArrangement}

\endmetadata

% this is the default PlanetMath preamble.  as your knowledge
% of TeX increases, you will probably want to edit this, but
% it should be fine as is for beginners.

% almost certainly you want these
\usepackage{amssymb}
\usepackage{amsmath}
\usepackage{amsfonts}

% used for TeXing text within eps files
%\usepackage{psfrag}
% need this for including graphics (\includegraphics)
%\usepackage{graphicx}
% for neatly defining theorems and propositions
\usepackage{amsthm}
% making logically defined graphics
%%\usepackage{xypic}

% there are many more packages, add them here as you need them

% define commands here
\DeclareMathOperator{\Span}{span}
\theoremstyle{definition}
\newtheorem*{example*}{Example}
\begin{document}
\PMlinkescapeword{dimension}
\PMlinkescapeword{structure}
Let $\mathcal{A}$ be a finite subspace arrangement in a 
finite-dimensional vector space $V$.
The {\em \PMlinkescapetext{intersection semilattice}} of 
$\mathcal{A}$ is the subspace
arrangement $L(\mathcal{A})$ defined by taking the 
\PMlinkname{closure}{ClosureAxioms} 
of $\mathcal{A}$ under intersections.  More formally, let
\[
L(\mathcal{A}) = \bigl\{ \bigcap_{H\in\mathcal{S}} H
                 \mid    \mathcal{S}\subset\mathcal{A}
                 \bigr\}.
\]
\PMlinkname{Order}{Poset} the elements of $L(\mathcal{A})$ 
by reverse inclusion,
and give it the structure of a join-semilattice by defining
$H\vee K=H\cap K$ for all $H$, $K$ in $L(\mathcal{A})$.
Moreover, the elements of $L(\mathcal{A})$ are naturally 
graded by codimension.  If $\mathcal{A}$
happens to be a central arrangement, its intersection
semilattice is in fact a lattice, with the meet operation
defined by $H\wedge K=\Span(H\cup K)$, where 
$\Span(H\cup K)$ is the subspace of $V$ spanned by 
$H\cup K$.

% This will need a figure, which I will add soon.
%\begin{example*}
%Let $V=\mathbb{R}^4$ and let $\mathcal{A}=\{x_i=x_j\mid 1\le i<j\le 4\}$.
%Elements of the intersection lattice $L(\mathcal{A})$ of codimensio
%\end{example*}
%%%%%
%%%%%
\end{document}
