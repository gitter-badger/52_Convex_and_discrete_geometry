\documentclass[12pt]{article}
\usepackage{pmmeta}
\pmcanonicalname{Zonotope}
\pmcreated{2013-03-22 15:47:20}
\pmmodified{2013-03-22 15:47:20}
\pmowner{mps}{409}
\pmmodifier{mps}{409}
\pmtitle{zonotope}
\pmrecord{7}{37749}
\pmprivacy{1}
\pmauthor{mps}{409}
\pmtype{Definition}
\pmcomment{trigger rebuild}
\pmclassification{msc}{52B99}
\pmsynonym{zonohedron}{Zonotope}
\pmsynonym{zonohedra}{Zonotope}
\pmrelated{HyperplaneArrangement}

\endmetadata

% this is the default PlanetMath preamble.  as your knowledge
% of TeX increases, you will probably want to edit this, but
% it should be fine as is for beginners.

% almost certainly you want these
\usepackage{amssymb}
\usepackage{amsmath}
\usepackage{amsfonts}

% used for TeXing text within eps files
%\usepackage{psfrag}
% need this for including graphics (\includegraphics)
%\usepackage{graphicx}
% for neatly defining theorems and propositions
%\usepackage{amsthm}
% making logically defined graphics
%%%\usepackage{xypic}

% there are many more packages, add them here as you need them

% define commands here
\begin{document}
A \emph{zonotope} is a polytope which can be obtained as the
\PMlinkname{Minkowski sum}{MinkowskiSum3} of finitely many 
closed line segments in $\mathbb{R}^n$.  Three-dimensional zonotopes are also sometimes called \emph{zonohedra}.  Zonotopes are dual to finite hyperplane arrangements.  They are centrally symmetric, compact, convex sets.

For example, the unit $n$-cube is the Minkowski sum of
the line segments from the origin to the standard unit vectors $e_i$ 
for $1\le i\le n$.  
A hexagon is also a zonotope; for example, the Minkowski 
sum of the line segments based at the origin with endpoints at $(1,0)$, $(1,1)$, and $(0,1)$ is a hexagon.  In fact, any projection of an $n$-cube is a zonotope.

The prism of a zonotope is always a zonotope, but the pyramid of a 
zonotope need not be.  In particular, the 
\PMlinkname{$n$-simplex}{HomologyTopologicalSpace} is only a
zonotope for $n\le 1$.

\begin{thebibliography}{9}
\bibitem{cite:BER}
Billera, L., R. Ehrenborg, and M. Readdy, \emph{The $\mathbf{cd}$-index of zonotopes and arrangements}, in \emph{Mathematical essays in honor of Gian-Carlo Rota}, (B. E. Sagan and R. P. Stanley, eds.), Birkhäuser, Boston, 1998, pp. 23--40.
\bibitem{cite:Z}
Ziegler, G., \emph{Lectures on polytopes}, Springer-Verlag, 1997.
\end{thebibliography}
%%%%%
%%%%%
\end{document}
