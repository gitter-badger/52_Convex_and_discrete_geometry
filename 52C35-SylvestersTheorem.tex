\documentclass[12pt]{article}
\usepackage{pmmeta}
\pmcanonicalname{SylvestersTheorem}
\pmcreated{2013-03-22 13:59:36}
\pmmodified{2013-03-22 13:59:36}
\pmowner{bbukh}{348}
\pmmodifier{bbukh}{348}
\pmtitle{Sylvester's theorem}
\pmrecord{5}{34770}
\pmprivacy{1}
\pmauthor{bbukh}{348}
\pmtype{Theorem}
\pmcomment{trigger rebuild}
\pmclassification{msc}{52C35}
\pmclassification{msc}{51M04}

\usepackage{amssymb}
\usepackage{amsmath}
\usepackage{amsfonts}

% used for TeXing text within eps files
%\usepackage{psfrag}
% need this for including graphics (\includegraphics)
%\usepackage{graphicx}
% for neatly defining theorems and propositions
\usepackage{amsthm}
% making logically defined graphics
%%%\usepackage{xypic}

\makeatletter
\@ifundefined{bibname}{}{\renewcommand{\bibname}{References}}
\makeatother
\begin{document}
For every finite collection of non-collinear points in Euclidean space, there is a line that passes through exactly two of them.

\begin{proof}
Consider all lines passing through two or more points in the collection.
Since not all points lie on the same line, among pairs of points and lines that are non-incident we can find a point $A$ and a line $l$ such that the distance $d(A,l)$ between them is minimal. Suppose the line $l$ contained more than two points. Then at least two of them, say $B$ and $C$, would lie on the same side of the perpendicular from $p$ to $l$. But then either $d(AB,C)$ or $d(AC,B)$ would be smaller than the distance $d(A,l)$ which contradicts the minimality of $d(A,l)$.
\end{proof}
%%%%%
%%%%%
\end{document}
